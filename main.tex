\documentclass[12pt,twoside]{report}


%use texdoc xxx to view the document about the package.

%chinese support.
\usepackage{xeCJK}
%mathtype support.
\usepackage{amsmath}
\usepackage{amsthm}
%page setting support.
\usepackage{geometry}
%figure insertion support.
\usepackage{graphicx}
%color support
\usepackage{xcolor}
%code block support
\usepackage{listings}
%drawing support
\usepackage{tikz}
%head foot set support
\usepackage{fancyhdr}
% 设置章节格式
\usepackage{titlesec} 
%font support
\usepackage{fontspec} 
%set indent
\usepackage{indentfirst}
%support box
\usepackage{framed}

\usepackage{caption}
\usepackage{algorithm}
\usepackage{algcompatible} 
\usepackage{algorithmicx} 
\usepackage{algpseudocode}
% \usepackage[linesnumbered,boxed]{algorithm2e}  

\usepackage[bookmarks=true,colorlinks,linkcolor=black]{hyperref}




%%%%%%%%%%%%%%%%%%%%%%%%%%%%%%%%%%%%%set color%%%%%%%%%%%%%%%%%%%%%%%%%%%%%%%%%%%%%%%%%%%%%%%%%%%%%%%
\definecolor{code-comments}{RGB}{0,180,0}   
\definecolor{background}{RGB}{234,244,244}

%%%%%%%%%%%%%%%%%%%%%%%%%%%%%%%%%%%%%set theoremstyle%%%%%%%%%%%%%%%%%%%%%%%%%%%%%%%%%%%%%%%%%%%%%%%%
\newtheoremstyle{nrush-thm-style}% hnamei
{20pt}% Space above
{20pt}% hSpace below
{\itshape}% Body font
{0pt}% Indent amount
{\bfseries\itshape\large}% Theorem head font
{:}% Punctuation after theorem head
{1em}% Space after theorem head
{}% Theorem head spec (can be left empty, meaning ‘normal’)

\theoremstyle{nrush-thm-style}

\makeatletter
\renewenvironment{proof}[1][Proof]{{\bfseries\large\itshape\noindent #1:\ \ }}{\ \rule{0.5em}{0.5em}}
\makeatother

%%%%%%%%%%%%%%%%%%%%%%%%%%%%%%%%%%%%%set pagesize%%%%%%%%%%%%%%%%%%%%%%%%%%%%%%%%%%%%%%%%%%%%%%%%%%%%
\geometry{a4paper,left=2cm,right=2cm,top=2cm,bottom=2cm}
\geometry{a4paper,vscale=0.9,hscale = 0.9}

%%%%%%%%%%%%%%%%%%%%%%%%%%%%%%%%%%%%%set foot%%%%%%%%%%%%%%%%%%%%%%%%%%%%%%%%%%%%%%%%%%%%%%%%%%%%%%%%
\pagestyle{plain}
\lhead{}
\chead{}
\rhead{}
\lfoot{}
\cfoot{}
\rfoot{\thepage}

%%%%%%%%%%%%%%%%%%%%%%%%%%%%%%%%%%%%%set paragraph-style%%%%%%%%%%%%%%%%%%%%%%%%%%%%%%%%%%%%%%%%%%%%%
% \setmainfont{Kozuka Mincho Pro R}
\setlength{\parindent}{2em}

%%%%%%%%%%%%%%%%%%%%%%%%%%%%%%%%%%%%%set code-style%%%%%%%%%%%%%%%%%%%%%%%%%%%%%%%%%%%%%%%%%%%%%%%%%%
\lstset{
backgroundcolor = \color{background},
keywordstyle=\color{blue!100},
commentstyle=\color{code-comments},
basicstyle=\small\sffamily ,
language=Matlab,
breaklines=true,
captionpos=b,
frame = single,
}

%%%%%%%%%%%%%%%%%%%%%%%%%%%%%%%%%%%%%set algorithm-style%%%%%%%%%%%%%%%%%%%%%%%%%%%%%%%%%%%%%%%%%%%%%%
\renewcommand{\algorithmicrequire}{\textbf{Input:}}  % Use Input in the format of Algorithm  
\renewcommand{\algorithmicensure}{\textbf{Output:}} % Use Output in the format of Algorithm

\makeatletter
\newenvironment{breakablealgorithm}
  {% \begin{breakablealgorithm}
   \begin{center}
     \refstepcounter{algorithm}% New algorithm
     \hrule height.8pt depth0pt \kern2pt% \@fs@pre for \@fs@ruled
     \renewcommand{\caption}[2][\relax]{% Make a new \caption
       {\raggedright\textbf{\ALG@name~\thealgorithm} ##2\par}%
       \ifx\relax##1\relax % #1 is \relax
         \addcontentsline{loa}{algorithm}{\protect\numberline{\thealgorithm}##2}%
       \else % #1 is not \relax
         \addcontentsline{loa}{algorithm}{\protect\numberline{\thealgorithm}##1}%
       \fi
       \kern2pt\hrule\kern2pt
     }
  }{% \end{breakablealgorithm}
     \kern2pt\hrule\relax% \@fs@post for \@fs@ruled
   \end{center}
  }
\makeatother


%%%%%%%%%%%%%%%%%%%%%%%%%%%%%%%%%%%%%set tikz%%%%%%%%%%$%%%%%%%%%%%%%%%%%%%%%%%%%%%%%%%%%%%%%%%%%%%%%%
\usetikzlibrary{calc,matrix,decorations.markings,decorations.pathreplacing}
\usetikzlibrary{arrows,shapes,chains}

% 设置中文日期格式
\renewcommand{\today}{\number\year 年 \number\month 月 \number\day 日}
% 设置中文目录
\renewcommand{\contentsname}{目 录}
% 设置章节格式
\titleformat{\chapter}{\raggedright\Huge\bfseries}{第\,\thechapter\,章}{1em}{}






\input{theorem.tex}

\begin{document}
\title{\itshape{Shorthand Book}}
\author{Ji Youzhou}
\date{\today}
\maketitle
\thispagestyle{empty}
\newpage

\tableofcontents
\setcounter{page}{1}
\newpage

\chapter{C/C++}
\section{C++重载}
C++ 允许在同一作用域中的某个函数和运算符指定多个定义,分别称为函数重载和运算符重载。重载函数或符号的名称相同但其功能并不相同,编译器会根据使用的形式来确定是那一个操作。
运算符运算符重载格式
\begin{lstlisting}[language=C]
    <class-name> operator<operator-symbol>(inputs);
\end{lstlisting}
并非所有的运算符都可以重载,部分运算符如:
\begin{lstlisting}[language=C]
    .
    .*
    ->*
    ::
    ?:
    #
    sizeof
\end{lstlisting}
是不可以被重载的。

\chapter{Data structure}
\section{二叉树的前,中,候序遍历(DLR,LDR,LRD)的含义}
\begin{itemize}
    \item DLR: 根节点,左节点,右节点遍历。
    \item LDR: 左节点,根节点,右节点遍历。
    \item LRD: 左节点,右节点,根节点遍历。
\end{itemize}

\begin{figure}[H]
    \centering  
	\includegraphics[width=0.3\linewidth]{pic/tree}  
	\caption{Tree}  
\end{figure}

\begin{itemize}
    \item DLR: 1 2 4 5 7 8 3 6
    \item LDR: 4 2 7 5 8 1 3 6
    \item LRD: 4 7 8 5 2 6 3 1
\end{itemize}

得到一颗树的LDR和DLR或LRD中的一种是可以获得唯一的树的,如果仅仅是DLR和LRD则无法唯一确定。

\chapter{Opencv}
\section{VS环境下鼠标移到Mat卡死问题}
Mat注释过长,转到Mat定义处(鼠标不停留选中),在Mat与注释间加一空行保存。

\section{分离RGB三通道}

\begin{lstlisting}[language=C]
    \\ 分配地址
    vector<Mat> bgr;
    \\ 分割窗口
    split(src,bgr); 
\end{lstlisting}

\section{图片显示过大问题}

\begin{lstlisting}[language=C]
    \\ 设置窗口可调整
    namedWindow("src", WINDOW_NORMAL); 
    \\ 调整窗口大小
    cvResizeWindow("src", 500, 500);
    imshow("src", src);   
\end{lstlisting}

\section{VS下,使用nameWindow函数,imshow时却出项两个同名窗口问题}

在\textbf{项目属性 $\rightarrow$ 链接器 $\rightarrow$ 输入 $\rightarrow$ 附加依赖项}中去掉两个\textbf{opencv\_worldxxx.lib}两个文件中的一个,debug时去掉不带d的。

\section{遍历Mat元数}
Mat::forEach()具有最高的效率,对于Mat::at和指针来说其实两者效率差不都,指针略优。

\chapter{杂项}
\setcounter{page}{1}
\section{半收敛算法}
所谓半收敛算法直观印象就是说,算法误差降低到某一低点后,又回升震荡,对判断何时真正停止算法照成了困扰。

\section{椭圆的通径与偏心率}
所谓通径,就是过焦点作垂直于长轴的直线,其与椭圆交于两点,这条线段的长度就叫做通径。偏心率指的是椭圆焦距与其长轴长度的比值。
\end{document}